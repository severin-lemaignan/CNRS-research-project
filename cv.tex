%\chapter{Academic track-record and contributions}\label{cv}

\section{Academic profile}\label{early-achievements-track-record}


Since I completed my joint PhD in Cognitive Robotics from the CNRS/LAAS (France) and the
Technical University of Munich (Germany), for which I received the GdR Robotique \emph{Best
PhD in Robotics 2012} award from French CNRS and the prized \emph{Cumma Summa
Laude} distinction in Germany, I have emerged as a rising leader in
HRI.

Soon after my PhD, I created and successfully led for 2 years a HRI research
group within the AI for Learning CHILI Lab at EPFL (Switzerland). While my
original training was in \textbf{symbolic cognition \& AI for autonomous
robotics}, my postdoctoral stay at the highly cross-disciplinary CHILI Lab gave
me the opportunity to become an expert in \textbf{child-robot interaction} and
\textbf{robotics for learning}, while providing me with a solid footing in
\textbf{experimental sciences, socio-psychology and education sciences}.

I was then awarded an EU \textbf{H2020 Marie Sklodovska-Curie Individual
Fellowship} and I engaged in basic research on artificial cognition at the
University of Plymouth, UK: over 2 years, I explored the \textbf{underpinnings
of artificial social cognition}. I \textbf{contributed significantly to the
framing of the emerging field of data-driven HRI}, also releasing the PInSoRo
open dataset~\cite{pinsoro2018}, a
one-in-a-kind dataset of natural child-child and child-robot social
interactions.

I join the Bristol Robotics Lab (BRL, largest co-located robotic lab in the UK)
in 2018, first as a Senior Researcher, and since 2019, as a permanent
\textbf{Associate Professor in Social Robotics and AI}. I am \textbf{in charge
of defining and implementing the lab's research strategy in human-robot
interactions}, and my field of expertise covers \textbf{the socio-cognitive
aspects of human-robot interaction, from the perspective of human cognition,
social signal processing and the design and implementation of cognitive
architectures for robots}. I focus my
\textbf{experimental work on real-world, natural human-robot interactions}, with
a particular interest on \textbf{child-robot interactions in educative
settings}, exploring how robots can support teachers and therapists to develop
engaging novel learning paradigms.


{\bf I detail my main scientific and technical contributions to date in the
\emph{Scientific and technical contributions} section, page~\pageref{sci-contribs}.}


\subsection{Significant awards}

\begin{tabular}{p{0.17\linewidth}p{0.8\linewidth}}
    \bf HRI'2017  & Best Paper award\\
    \bf HRI'2016  & Best Paper award\\
    \bf AAAI'2015  & Best Video award in Artificial Intelligence\\
    \bf HRI'2014  & Best Late Breaking Report award\\
    \bf 2012         & GdR Robotique {\bf Best PhD in Robotics 2012} award, CNRS, France \\
    \bf 2012         & PhD with {\bf High Distinction} (“Summa Cum Laude”), TU Munich\\
    \bf Ro-Man'2010  & Best paper award\\
\end{tabular}

\subsection{Significant fellowships and grants}

\begin{tabular}{p{0.17\linewidth}p{0.8\linewidth}}
    \bf 2020 & \textbf{Submission of an ERC Consolidator} fellowship (unsuccessful) \\
    \bf 2020 & PI \textbf{robots4SEN} project, UWE VC Grand Challenges, £30K
               \newline \emph{$\rightarrow$ deployment of autonomous robots in a school for autistic
               children}\\
    \bf 2019 & UWE Vice Chancellor Accelerator Fellowship \\
    \bf 2018 & Co-I \textbf{CAV Forth} project, Innovate UK, £5M 
               \newline \emph{$\rightarrow$ first paying-service deployment of an
               autonomous bus in Scotland}\\
    \bf 2015 -- 2017 & {\bf EU Marie Skłodowska-Curie Individual Fellowship}
    \newline \emph{$\rightarrow$ Theory of Mind and social robotics, Plymouth University, UK} \\
\end{tabular}

\vspace{2em}
\section{Contributions to the development of individuals}

While at the `AI for Learning' CHILI Lab at EPFL, I created and successfully
led for 2 years the HRI resesarch group, supervising in total
10 students (including 4 PhD students with whom I co-authored a total of 18
papers). Within that short timeframe, I established CHILI as an internationally
recognised research lab in robotics for education. 

Then, during my EU Marie Skłodowska-Curie post-doc at Plymouth University, I
further co-supervised 3 PhD students (co-authoring 17 publications with them).

My current role as a permanent \textbf{Associate Professor in Social Robotics
and AI} at the Bristol Robotics Laboratory (BRL, largest co-located robotic lab
in the UK) recognise my leadership. I am \textbf{in charge of defining and
implementing the lab's research strategy in human-robot interactions}. I created
the Embedded Cognition for Human-Robot Interactions (ECHOS) research group, that
I now co-lead, supervising 15+ PhDs and post-docs. I also supervise the BRL's
Connected Autonomous Vehicles research group (5 students and post-docs).
Specifically, the ECHOS group covers most aspects of situated AI for human-robot
interaction, \textbf{my role includes strategic planning of the group
activities, scientific guidance, recruitment of staff and prospective students,
and grant applications}.

\subsection{Supervision of graduate students and postdoctoral fellows}

\begin{tabular}{p{0.17\linewidth}p{0.8\linewidth}}
    \bf 2018 -- 2019 & \textbf{2 post-docs}, \textbf{5 PhDs}, \textbf{4 MSc students}, Bristol Robotics Lab, UWE, UK \\
    \bf 2015 -- 2018 & \textbf{3 PhDs}, Plymouth University, UK \\
    \bf 2013 -- 2015 & \textbf{5 PhDs}, \textbf{5 MSc students}, EPFL, Switzerland \\
    \bf 2012 -- 2013 & \textbf{2 MSc students}, LAAS-CNRS, France \\
\end{tabular}


\subsection{Teaching activities}

\begin{tabular}{p{0.17\linewidth}p{0.8\linewidth}}
    \bf 2019 --  & \textbf{Associate Professor} teaching at postgraduate level, UWE, UK \\
    \bf 2018 -- 2019 & \textbf{Senior Lecturer} teaching at postgraduate level, UWE, UK \\
    \bf 2015 -- 2018 & \textbf{Lecturer} teaching at undergraduate \&
    postgraduate levels (robotics fundamentals, software engineering, human-robot interaction), Plymouth University, UK \\
    \bf 2013 -- 2015 & \textbf{Teaching Assistant} teaching at undergraduate level (Visual Computing), EPFL, Switzerland \\
    \bf 2008 -- 2012 & \textbf{Teaching Assistant} teaching at undergraduate level (programming, databases, ontologies), INSA Toulouse, France \\
\end{tabular}

\vspace{2em}
\section{Contributions to the wider research community}

Since my PhD, I have established strong peer recognition in the field of human-robot interaction
and cognitive robotics. This includes:

\begin{itemize}[noitemsep,topsep=0pt,parsep=0pt,partopsep=0pt]
    \item numerous \textbf{invited talks} at national and international symposiums and
        events (9 invited talks since Jan. 2018, including \textbf{keynotes} at the UK Robotics
and Autonomous Systems 2019 conference, and at the 2018 AAAI Fall Symposium);
    \item invited to \textbf{high-profile editorial roles}: Programme Committee member of the HRI
conference since 2015; editor of Frontiers In Robotics and AI journal; editor or
Programme Committee member of several leading conferences in AI and Robotics
        (RSS, IROS, IJCAI, HAI, AAMAS);
    \item invited member of the UK EPSRC Peer Review College; member of the EU
        H2020 peer review college; invited reviewer for the French, Dutch, Israeli research agencies;
    \item active role (organisation committee and/or programme committee in
        major conferences in robotics and AI (eg IEEE IROS, RSS, IEEE/ACM HRI, IJCAI);
    \item six invitations to PhD defense committees over the last two years.
\end{itemize}


\subsection{Organisation of scientific meetings}

\begin{tabular}{p{0.1\linewidth}p{0.85\linewidth}}
    \bf 2021 & \textbf{ACM/IEEE Human-Robot Interaction conference}, Student
    Design Competition chair, virtually held \\
    \bf 2020 & \textbf{ACM/IEEE Human-Robot Interaction conference}, 700+ participants, local chair, Cambridge, UK \\
    \bf 2017 & \textbf{ACM/IEEE Human-Robot Interaction conference}, 400+
    participants, alt.HRI chair, Vienna, AT \\
    \bf 2016 & \textbf{2nd Intl. workshop on Cognitive Architecture for Social HRI}, 45 participants, programme chair, Christchurch, NZ \\
    \bf 2014 & \textbf{Intl. workshop on Simulation for HRI}, 35 participants, programme chair, Bielefeld, DE \\
    \bf 2012 & \textbf{Intl. workshop on MORSE and its applications}, 30 participants, programme chair, Toulouse, FR \\
    \bf 2009 & \textbf{Cognitive Sciences’ Young Researchers Conference}, 150 participants, steering committee, Toulouse, FR \\
\end{tabular}

\subsection{Institutional responsibilities}

\begin{tabular}{p{0.17\linewidth}p{0.8\linewidth}}
    \bf 2019 -- & Full member of the EPSRC Peer Review college \\
    \bf 2017-- & EU H2020 member on the Peer Review College \\
    \bf 2019 -- & Head of the Outreach cluster, Faculty of Technology and Environment, UWE, UK \\
    \bf 2019-- & Invited PhD committee examiner (Örebro U., Uppsala U., KTH,
    Bielefeld U., LAAS-CNRS, BRL)\\
    \bf 2018 -- & HRI module co-lead, MSc level, University of the West of England, UK  \\
    \bf 2017 -- 2018 & Module leader, Robotics fundamentals (undergraduate level), University of Plymouth, UK \\
\end{tabular}

\subsection{Editorial activities}

\begin{tabular}{p{0.17\linewidth}p{0.8\linewidth}}
    \bf 2019 --  & Member of the Robotics, Science and System (RSS) Programme Committee  \\
    \bf 2018 --  & Editorial board of \emph{Frontiers in AI and Robotics} \\
    \bf 2017 --  & Member of the IJCAI Programme Committee  \\
    \bf 2015 -- 2020 & Member of the IEEE/ACM HRI Programme Committee \\
    \bf 2017 -- 2019 & Member of the IEEE IROS Programme Committee  \\
    \bf 2017 -- 2018 & Member of the HAI Programme Committee  \\
\end{tabular}

\vspace{2em}
\section{Contributions to the broader society}

I \textbf{actively engage with policy makers, at national and European
level}: for instance, over the past 2 years, I have been directly interacting
(through participating to panels, visits and one-to-one discussions) with the EU
Research Executive Agency (MSCA AI Cluster 2019); the UK minister for Business,
Energy and Industrial Strategy Greg Clark; the UK minister for Universities,
Science, Research and Innovation Chris Skidmore; the chair of the West of
England authority Tim Bowles; the UK Research \& Innovation Portfolio
manager for Robotics Clara Morri.

I have a \textbf{strong track record of tech transfer}, through patenting (US patent
US20190016213A1) and involvement in national (UK) and EU-level projects focused on
tech-transfer (InnovateUK ROBOPILOT, CAPRI, CAVForth; EU Terrinet, SABRE).

Finally, I actively engage in \textbf{research communication}: my past research has been
covered several times by mainstream international media, including press
releases by Reuters, Press Association; TV coverage by the BBC, Sky News; radio
interviews and broadcast. My academic website (\url{academia.skadge.org})
showcases this media coverage. I also maintain an active, science-focused,
presence on the social media (Twitter handle: @skadge).

\subsection{Policy making}

\begin{tabular}{p{0.17\linewidth}p{0.8\linewidth}}
    \bf 2020 -- & {\bf Expert Collaborator for the European Joint Research Centre} contributing to the UNICEF Guidelines for Responsible Child-Robots Interactions \\
    \bf 2019  & {\bf Invited panel by the EU Research Executive Agency} at the 2019 MSCA AI Cluster, sharing expertise in Human-Robot Interaction \\
\end{tabular}

\subsection{Technology transfer}
\begin{tabular}{p{0.17\linewidth}p{0.8\linewidth}}
    \bf 2018 -- & Co-I on UKRI InnovateUK projects ROBOPILOT, CAPRI, CAVForth, involving direct transfer of technology for automated verification of autonomous vehicles \\
    \bf 2018 -- & Scientific advisor for KickSum Ltd., in the frame of the EU-funded SABRE project \\
    \bf 2018  & Co-inventor on US patent US20190016213A1 on back-driveable, haptic locomotion for small robots \\
\end{tabular}

\subsection{Selected outreach and public dissemination}

\begin{tabular}{p{0.17\linewidth}p{0.8\linewidth}}

    \bf 2019--& Cluster Lead for STEM outreach, University of the West of England \\
    \bf 2019--& Scientific advisor for the Bristol's Science Centre \\
    \bf 2019 & Hosted \href{https://share.coveragebook.com/b/6d7defc7c4a49e93}{large media event} for the Couch25K study~\cite{winkle2020insitu} \\
    \bf 2016--& UK \& EU Robotics Weeks coordinator, University of Plymouth, University of the West of England \\
    \bf 2015 & Hosted large media event for the CoWriter study~\cite{lemaignan2016learning} (coverage by Reuters, BBC Arabic, FastCompany) \\
    \bf 2011& 'Roboscopie' Human-Robot public theater performance, Science Day'11 \url{http://bit.ly/1LQpNWA} \\
    \bf 2008--2011& Toulouse's Cognitive Sciences Students Association, Co-chair \\
    \bf 1997--2012& Executive Committee \& Head of Educational Robotics, Planète Sciences (including coordination of the \textit{EUROBOT} Robotic Competition) \\

\end{tabular}
\vspace{2em}


\section{Scientific and technical contributions}
\label{sci-contribs}

This expertise is recognised internationally: I have a substantial track record
of academic outputs. Since 2008, I have authored or co-authored \textbf{75+
peer-reviewed publications} in international journals and conferences, leading
to \textbf{2700+ citations}, h-index of 26, i10-index of 43 (source: Google
Scholar).



My research activity in robotics and human-robot interaction started with my PhD
in 2008. Since then, my scientific journey


I am also a technology expert, with major software and hardware contributions to
the robotic community (including contributions to OpenCV and core components of
Robot Operating System, ROS). As such, I have a clear grasp of the technical
feasibility of the proposed work. I am also in the rare position of having
substantial experience in designing and running full architectures for complex
autonomous social
robots~\cite{lemaignan2017artificial,winkle2020insitu}.



\subsection{Technical contributions}

Since 2010, I have made a number of significant technical contribution to the
field. I have always adopted a open-science approach, releasing all of the
software contribution to the wider community.

I list hereafter the most significant software packages (typically the one with an
associated publication), followed by additional noteworthy technical
contributions.

\begin{itemize}
    \item the \texttt{oro} knowledge base~\autocite{lemaignan2010oro} --
        this high-cited work introduced the usage of ontologies (and linked techniques like semantic
        reasoning) in robotics.

    \item the natural language processing with semantic grounding tool
        \texttt{dialogs}~\autocite{lemaignan2011grounding} -- this other
        highly-cited tool demonstrated how natural language and interactive
        semantic learning could be realised by combining semantic reasoning with
        advanced human perception.

    \item the \texttt{MORSE} simulator~\autocite{echeverria2011morse,
        lemaignan2012morse}, one of the very first simulator enabling
        human-robot interaction simulation, and used by tenths of universities
        worldwide since its inception.

    \item the GenoM verifiable software module
        generator~\autocite{mallet2010genom3} -- this tool makes it possible to
        abstractly specify a robotic module, and automatically generate a code
        skeleton whose behaviour can be proven correct.

    \item the Python-based \texttt{pyRobots} asynchronous supervision
        framework~\autocite{lemaignan2015pyrobots} -- adapted some of the concepts
        orginally created in the URBI language to Python, making it possible to
        easily write asynchronous supervisors for robots using eg ROS.

    \item integration of the LAAS architecture for social
        robots~\autocite{lemaignan2017artificial} -- I coordinated the effort of
        a large team of researchers at LAAS to integrate a significant number of
        software modules in a coherent architecture for social interaction. One
        of my most-cited paper.

    \item a high-accuracy 2D localisation method based on structured
        patterns~\autocite{hostettler2016realtime} -- I supervised this work
        in which we attempted to address the difficult issue of high-accuracy
        indoor localisation in complex, highly-occluded environment. Our method,
        which relies on decoding structured patterns placed in the environment,
        allows for sub-mm localisation with very low computational cost (can
        fully run on a microcontroller)

    \item the 3D situation assessment platform
        \texttt{underworlds}~\autocite{lemaignan2018underworlds} -- this tool
        is a distributed scene-graph, making it possible to maintain a joint
        dynamic 3D model of environment across software modules. It features
        sensor fusion, and spatial reasoning capabilities like perspective
        taking.

    \item a new algorithm for interactive reinforcement
        learning~\autocite{senft2017supervised} -- the algorithm, developed by
        one of my student, has enabled for the first to to teach a robot both a
        task and a social action policy \emph{while being in use in the field}.
        We were able to show that after a short training phase, the robot was
        able to reach fully autonomy on a complex educative task.
\end{itemize}

     a review of object recognition
techniques~\autocite{wallbridge2017qualitative}, 

\begin{itemize}
    \item Python3 ROS port
    \item initial port of the Nao robot to ROS
    \item port of the HOAP-3 humanoid robot to ROS
\end{itemize}


\subsection{Selected scientific outputs}

\resizebox{\linewidth}{!}{
\hspace*{-0.5cm}\begin{tabular}{p{1.7cm}p{7cm}p{8cm}}

    \vspace{-0.2cm}\includegraphics[height=2.2cm]{thumbs/2019-science.png} & Senft, E.,
    \ul{Lemaignan, S.}, Baxter, P., Bartlett, M., Belpaeme, T.
    \newline\href{https://doi.org/10.1126/scirobotics.aat1186}{\textbf{Teaching robots
    social autonomy from in situ human guidance}}
    \newline \textit{Science Robotics} 2019
    & \small A novel human-in-the-loop machine learning approach
    to implement social autonomy in a robot, with several deployments in UK
    public schools. This is a first-in-kind demonstration of learning autonomous
    action policy in a high dimensional, socially complex,
    environment.\textbf{\newline[main study supervisor]} \\


    \vspace{-.20cm}\includegraphics[height=2.2cm]{thumbs/2019-frontiers-chris.jpg} &

    Wallbridge, C., \ul{Lemaignan, S.}, Senft, E., Belpaeme, T.  
    \newline\href{https://doi.org/10.3389/frobt.2019.00067}{\textbf{Generating
    Spatial Referring Expressions in a Social Robot: Dynamic vs Non-Ambiguous}}
    \newline \textit{Frontiers in AI and Robotics} 2019
    & \small Challenges the common understanding that robots should be
    unambiguous: we show that ambiguity is often desirable for fluid and natural
    human-robot interactions.\textbf{\newline[main study supervisor]}  \\

    \vspace{-.20cm}\includegraphics[height=2.2cm]{thumbs/2019-frontiers-maddy.jpg} &

    Bartlett, M., Edmunds, C. E. R., Belpaeme, T., Thill, S., \ul{Lemaignan, S.} 
    \href{https://doi.org/10.3389/frobt.2019.00049}{\textbf{What Can You See? Identifying Cues on Internal States from the
    Kinematics of Natural Social Interactions}} 
    \newline \textit{Frontiers in AI and Robotics} 2019
    & \small Investigates how partially hidden `internal states' (like emotions,
    cooperativeness, etc) can be decoded from simple visible cues, like
    skeletons. Also demonstrates that social situations can be described along 3
    simple dimensions.\textbf{\newline[main study supervisor]}\\



    \vspace{-.20cm}\includegraphics[height=2.2cm]{thumbs/2018-plosone.jpg} &

    \ul{Lemaignan, S.}, Edmunds E. R., C., Senft, E., Belpaeme, T.
    \newline\href{https://doi.org/10.1371/journal.pone.0205999}{\textbf{The
    PInSoRo dataset: Supporting the data-driven study of child-robot social
    dynamics}}
    \newline \textit{PLOS ONE} 2018
    & \small A first-in-kind, large scale dataset of child-child and child-robot social interactions. Design
    with machine learning in mind, this dataset effectively opens up the field
    of data-driven social psychology, with direct applications in AI and social
    robotics.\textbf{[principal investigator]}\\
\end{tabular}
}

\resizebox{\linewidth}{!}{
\hspace*{-0.5cm}\begin{tabular}{p{1.7cm}p{7cm}p{8cm}}

    \vspace{-.20cm}\includegraphics[height=2.2cm]{thumbs/2018-underworlds.jpg} &

    \ul{Lemaignan, S.}, Sallami, Y., Wallbridge, C., Clodic, A., Alami,
    R. 
   \newline\href{https://doi.org/10.1109/IROS.2018.8594094}{\textbf{\sc
    underworlds: Cascading Situation Assessment for Robots}}
    \newline\textit{IEEE IROS} 2018

    & \small A novel representation technique to efficiently
    represent multiple parallel states of the world, including imaginary ones.
    This ability is critical to represent spatio-temporal predictions, and to
    create models of other agents' representations.
    \textbf{[principal investigator]}\\



    \vspace{-.20cm}\includegraphics[height=2.2cm]{thumbs/2017-sparc.jpg} &

    Senft, E., Baxter, P., Kennedy, J., \ul{Lemaignan, S.}, Belpaeme, T.
    \newline\href{https://doi.org/10.1016/j.patrec.2017.03.015}{\textbf{Supervised
    Autonomy for Online Learning in Human-Robot Interaction}}
    \newline \textit{Pattern Recognition Letters} 2017
    & \small The mathematical and technical bases of the SPARC
    paradigm for human-in-the-loop machine learning, showing that
    high-dimensional problems can be learnt effectively and rapidely thanks to
    an innovative input feature selection mechanism.
    \textbf{\newline[student supervisor; 22 citations]}\\


    \vspace{-.20cm}\includegraphics[height=2.2cm]{thumbs/2017-ai-cover.jpg} &

    \ul{Lemaignan, S.}, Warnier, M., Sisbot, E.A., Clodic, A., Alami, R.
    \newline
    \href{https://doi.org/10.1016/j.artint.2016.07.002}{\textbf{Artificial
    Cognition for Social Human-Robot Interaction: An Implementation}}
    \newline \textit{Artificial Intelligence} 2017
    & \small Landmark article: one of the first complete, semantic-aware, robotic architecture for
    human-robot interaction, including symbolic knowledge representation,
    situation assessment, natural language grounding, task planning, human-aware
    motion planning and execution. \textbf{\newline[principal investigator and
    coordinator; 143 citations]}\\


    \vspace{-.20cm}\includegraphics[height=2.2cm]{thumbs/2016-cowriter.jpg} &

    \ul{Lemaignan, S.}, Jacq, A., Hood, D., Garcia, F., Paiva, A., Dillenbourg, P.
    \newline
    \href{https://doi.org/10.1109/MRA.2016.2546700}{\textbf{Learning by
    Teaching a Robot: The Case of Handwriting}}
    \newline \textit{Robotics and Automation Magazine} 2016
    & \small Long-term studies with children and
    therapists, where we \emph{reverse} the social role of the
    robot to significantly improve the children' self-confidence. A landmark in
    social robotics for education. \textbf{\newline[principal investigator; 141
    citations} (incl. conf. article) \textbf{]}\\



    \vspace{-.20cm}\includegraphics[height=2.2cm]{thumbs/2012-grounding.jpg} &

    \ul{Lemaignan, S.}, Ros, R., Sisbot, E. A., Alami, R., Beetz M.
    \href{https://doi.org/10.1007/s12369-011-0123-x}{\textbf{Grounding
    the Interaction: Anchoring Situated Discourse in Everyday Human-Robot
    Interaction}} 
    \newline \textit{Intl Journal of Social Robotics} 2012

    & \small In this paper, I show how symbolic knowledge representation can be
    used by robot to ground natural language interactions, also taking into
    account the unique perspective of the human interactor.
    \textbf{\newline[principal investigator; 100 citations]}\\

    \vspace{-.20cm}\includegraphics[height=2.2cm]{thumbs/2010-oro.jpg} &
    \ul{Lemaignan, S.}, Ros, R., Mösenlechner, L., Alami, R., Beetz, M.
    \newline\href{https://doi.org/10.1109/IROS.2010.5649547}{\textbf{ORO, a Knowledge Management Module for Cognitive Architectures in
    Robotics}}
    \newline \textit{IEEE IROS} 2010

    & \small One of the very first knowledge base designed and
    integrated in service robots. Pioneering work which played a key role in
    understanding how intelligent robot can represent their
    knowledge to facilitate communication with humans.
    \textbf{\newline[principal investigator; 158 citations]}\\

\end{tabular}
}

Table~\ref{pi-expertise} lists, per domain, some of my academic outputs that are
directly relevant to the research project.

%\begin{savenotes}
\begin{table}[h!]
    \centering
    \caption{\small PI's domains of expertise relevant to the research project}
    \begin{tabular}{rp{0.6\linewidth}}
        \toprule
        %\bf Expertise domain                  & \bf Corresponding publications by PI          \\
        %\midrule
        \textbf{Psycho-social underpinnings of HRI} \\  
        human factors & \small anthropomorphism\cite{lemaignan2014dynamics}, cognitive
        correlates\cite{lemaignan2014cognitive}, social influence\cite{winkle2019effective} \\
        trust, engagement, social presence & \small \cite{flook2019impact}\cite{lemaignan2015youre}\cite{fink2014which}\cite{irfan2018social}\cite{wijnen2020performing} \\
        theory of mind & \small perspective taking\cite{ros2010which, warnier2012when}, social mutual modelling\cite{lemaignan2015mutual,dillenbourg2016symmetry} \\
        \midrule
        \textbf{Social signal processing}\\
        non-verbal behaviours & \small attention\cite{lemaignan2016realtime},
        child-child dataset\cite{lemaignan2018pinsoro}, internal state decoding\cite{bartlett2019what} \\
        verbal interactions & \small speech recognition\cite{kennedy2017child}, dialogue grounding\cite{lemaignan2011grounding} \\
        \midrule
        \textbf{Behaviour generation} \\
        social behaviours & \small \cite{lallee2011towards}, verbal interactions\cite{wallbridge2019generating, wallbridge2019towards}, physical interactions\cite{gharbi2013natural} \\
        interactive reinforcement learning & \small
        \cite{senft2017leveraging,senft2017supervised, senft2019teaching,  winkle2020insitu} \\
        \midrule
        \textbf{Socio-cognitive architectures} \\
        architecture design & \small \cite{lemaignan2017artificial, baxter2016cognitive,lemaignan2014challenges,lallee2012towards, mallet2010genom3} \\
        knowledge representation & \small
        ontologies~\cite{lemaignan2010oro, lemaignan2013explicit} \\
        spatio-temporal modelling & \small object
        detection~\cite{wallbridge2017qualitative}, physics-aware situation
        assessment\cite{lemaignan2018underworlds,sallami2019simulation} \\
        \midrule
        \textbf{Fieldwork in HRI} & \small in
        classrooms~\cite{hood2015when, lemaignan2016learning, jacq2016building,
        baxter2015wider,kennedy2016cautious,senft2018robots}, at
        home~\cite{mondada2015ranger}, in public spaces~\cite{winkle2020insitu}\\
        %\midrule
        %Robot hardware design for interaction & \small \cite{ozgur2017cellulo, hostettler2016realtime} \\
        \bottomrule
    \end{tabular}
    \label{pi-expertise}
\end{table}
%\end{savenotes}



%\section{MAJOR COLLABORATIONS}

%Name of collaborators, Topic, Name of Faculty/ Department/Centre, Name of University/ Institution/ Country

%%%%%%%%%%%%%%%%%%%%%%%%%%%%%%%%%%%%%%%%%%%%%%%%%%%%%%%%%%%%%%%%%%%%%%%%%%%%%%%%%%%%%%%%%%%%%%%%%%%%%%%%%%%%
%\section{}

%%%%%%%%%%%%%%%%%%%%%%%%%%%%%%%%%%%%%%%%%%%%%%%%%%%%%%%%%%%%%%%%%%%%%%%%%%%%%%%%%%%%%%%%%%%%%%%%%%%%%%%%%%%%
%\section{}

%%%%%%%%%%%%%%%%%%%%%%%%%%%%%%%%%%%%%%%%%%%%%%%%%%%%%%%%%%%%%%%%%%%%%%%%%%%%%%%%%%%%%%%%%%%%%%%%%%%%%%%%%%%%
%\section{}

