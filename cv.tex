%\chapter{Academic track-record and contributions}\label{cv}

\section{Academic profile}\label{early-achievements-track-record}


Since I completed my joint PhD in Cognitive Robotics from the CNRS/LAAS (France) and the
Technical University of Munich (Germany), for which I received the GdR Robotique \emph{Best
PhD in Robotics 2012} award from French CNRS and the prized \emph{Cumma Summa
Laude} distinction in Germany, I have emerged as a rising leader in
HRI.

Soon after my PhD, I created and successfully led for 2 years a HRI research
group within the AI for Learning CHILI Lab at EPFL (Switzerland). While my
original training was in \textbf{symbolic cognition \& AI for autonomous
robotics}, my postdoctoral stay at the highly cross-disciplinary CHILI Lab gave
me the opportunity to become an expert in \textbf{child-robot interaction} and
\textbf{robotics for learning}, while providing me with a solid footing in
\textbf{experimental sciences, socio-psychology and education sciences}.

I was then awarded an highly competitive \textbf{EU H2020 Marie Sklodovska-Curie Individual
Fellowship} to engage in basic research on artificial cognition at the
University of Plymouth, UK: over a 2 years period, I explored the \textbf{underpinnings
of artificial social cognition}. I \textbf{contributed significantly to the
framing of the emerging field of data-driven HRI}, also releasing the PInSoRo
open dataset~\cite{pinsoro2018}, a
one-in-a-kind dataset of natural child-child and child-robot social
interactions.

I joined the Bristol Robotics Lab (BRL, largest co-located robotics lab in the UK)
in 2018, first as a Senior Researcher, and since 2019, as a permanent
\textbf{Associate Professor in Social Robotics and AI}. I am \textbf{in charge
of defining and implementing the lab's research strategy in human-robot
interactions}, and my field of expertise covers \textbf{the socio-cognitive
aspects of human-robot interaction, from the perspective of human cognition,
social signal processing and the design and implementation of cognitive
architectures for robots}. I focus my
experimental work on \textbf{real-world, natural human-robot interactions}, with
a particular interest for \textbf{child-robot interactions in educative
settings}, exploring how robots can support teachers and therapists to develop
engaging novel learning paradigms.


\subsection{Significant awards}

\begin{tabular}{p{0.17\linewidth}p{0.8\linewidth}}
    \bf HRI'2017  & Best Paper award\\
    \bf HRI'2016  & Best Paper award\\
    \bf AAAI'2015  & Best Video award in Artificial Intelligence\\
    \bf HRI'2014  & Best Late Breaking Report award\\
    \bf 2012         & GdR Robotique {\bf Best PhD in Robotics 2012} award, CNRS, France \\
    \bf 2012         & PhD with {\bf High Distinction} (“Summa Cum Laude”), TU Munich\\
    \bf Ro-Man'2010  & Best paper award\\
\end{tabular}

\subsection{Significant fellowships and grants}

\begin{tabular}{p{0.17\linewidth}p{0.8\linewidth}}
    \bf 2020 & \textbf{Submission of an ERC Consolidator} fellowship (unsuccessful) \\
    \bf 2020 & PI \textbf{robots4SEN} project, UWE VC Grand Challenges, £30K
               \newline \emph{$\rightarrow$ deployment of autonomous robots in a school for autistic
               children}\\
    \bf 2019 & UWE Vice Chancellor Accelerator Fellowship \\
    \bf 2018 & Co-I \textbf{CAV Forth} project, Innovate UK, £5M 
               \newline \emph{$\rightarrow$ first deployment of an
               autonomous bus service in Scotland}\\
    \bf 2015 -- 2017 & {\bf EU Marie Skłodowska-Curie Individual Fellowship}
    \newline \emph{$\rightarrow$ Theory of Mind and social robotics, Plymouth University, UK} \\
\end{tabular}

\vspace{2em}

\section{Scientific and technical contributions}
\label{sci-contribs}


\begin{framed}

    \vspace{1em}
\noindent\bf My main scientific contributions can be summarised as:

    \vspace{1em}
\begin{enumerate}
    \item a pioneering work on \emph{symbolic knowledge representation} and
        \emph{semantic-aware decisional architecture} for interactive robots;

    \item a key role in the development of \emph{situation assessment}, and more
        recently, \emph{social situation assessment}, as the formal
        investigation of the models required by robots to achieve autonomy in
        human environments;

    \item a key role in bridging research in cognitive psychology and sociology with robotics,
        with a number of cross-disciplinary literature surveys and experiments;

    \item a major contribution to the field of child-robot interaction, in
        particular by studying the importance of social engagement between
        a child and a robot;

    \item a leading role in the recent development of \emph{data-driven
        human-robot interaction}, via for instance the acquisition of large
        datasets, and the development of novel 'human-in-the-loop' machine
        learning algorithms.

\end{enumerate}

\vspace{1em}

\noindent In addition to these scientific contributions, I have led a number of
efforts on the methodology and technical tools used in robotics and human-robot
interaction through targeted publications and numerous open-source contributions.
This section contextualises all these contributions, and presents the impact of my
research through key scientific outputs.

    \vspace{1em}
\end{framed}

    \vspace{2em}



My scientific contributions cover three main research directions:
\textbf{Knowledge representation and cognitive architectures}, \textbf{Social
Robotics} and \textbf{Data-driven Human-Robot interactions}. I detail each of
those, as well as my main methodological and technical contributions, in the
following subsections. In addition, Table~\ref{pi-expertise} gives an overview
of my scientific fields of contribution, focusing on those which are directly
relevant to the research programme I propose to conduct at CNRS.

%\begin{savenotes}
\begin{table}[h!]
    \centering
    \caption{\small PI's domains of expertise relevant to the research project}
    \begin{tabular}{rp{0.6\linewidth}}
        \toprule
        %\bf Expertise domain                  & \bf Corresponding publications by PI          \\
        %\midrule
        \textbf{Psycho-social underpinnings of HRI} \\  
        human factors & \small anthropomorphism\autocite{lemaignan2014dynamics,lemaignan2014cognitive}, social influence\cite{winkle2019effective} \\
        trust, engagement, social presence & \small
        \autocite{flook2019impact,lemaignan2015youre,fink2014which,irfan2018social,wijnen2020performing} \\
        theory of mind & \small perspective taking\cite{ros2010which, warnier2012when}, social mutual modelling\cite{lemaignan2015mutual,dillenbourg2016symmetry} \\
        \midrule
        \textbf{Social signal processing}\\
        non-verbal behaviours & \small attention\cite{lemaignan2016realtime},
        child-child dataset\cite{lemaignan2018pinsoro}, internal state decoding\cite{bartlett2019what} \\
        verbal interactions & \small speech recognition\cite{kennedy2017child}, dialogue grounding\cite{lemaignan2011grounding} \\
        \midrule
        \textbf{Behaviour generation} \\
        social behaviours & \small \cite{lallee2011towards}, verbal interactions\cite{wallbridge2019generating, wallbridge2019towards}, physical interactions\cite{gharbi2013natural} \\
        interactive reinforcement learning & \small
        \cite{senft2017leveraging,senft2017supervised, senft2019teaching,  winkle2020insitu} \\
        \midrule
        \textbf{Socio-cognitive architectures} \\
        architecture design & \small \cite{lemaignan2017artificial, baxter2016cognitive,lemaignan2014challenges,lallee2012towards, mallet2010genom3} \\
        knowledge representation & \small
        ontologies~\cite{lemaignan2010oro, lemaignan2013explicit} \\
        spatio-temporal modelling & \small object
        detection~\cite{wallbridge2017qualitative}, situation
        assessment\cite{lemaignan2018underworlds,sallami2019simulation} \\
        \midrule
        \textbf{Fieldwork in HRI} & \small in
        classrooms~\cite{hood2015when, lemaignan2016learning, jacq2016building,
        baxter2015wider,kennedy2016cautious,senft2018robots}, at
        home~\cite{mondada2015ranger}, in public spaces~\cite{winkle2020insitu}\\
        %\midrule
        %Robot hardware design for interaction & \small \cite{ozgur2017cellulo, hostettler2016realtime} \\
        \bottomrule
    \end{tabular}
    \label{pi-expertise}
\end{table}
%\end{savenotes}


My research on these topics has resulted in a substantial track record of
academic outputs.  Since 2008, I have authored or co-authored \textbf{75+
peer-reviewed publications} in international journals and conferences, leading
to \textbf{2700+ citations}, h-index of 26, i10-index of 43
(\href{https://scholar.google.co.uk/citations?user=-CUOP2gAAAAJ}{source: Google
Scholar}).

\subsection{Knowledge representation and cognitive architectures}

My early research focused on investigating how high-level symbolic reasoning
could benefit human-robot interaction. My main insight was to bridge the
research performed in the Semantic Web community (that I had already researched
during my MSc~\cite{lemaignan2006mason}) with robotics, \textbf{integrating
ontologies into the robot's decisional
architecture}~\autocite[presented below]{lemaignan2010oro}. This highly-cited work, that I
conducted between LAAS-CNRS (Pr. Alami) and the Technical University of
Munich (Pr. Beetz), has had a large impact in AI for interactive robots,
\textbf{bridging low-level robot perceptions and commands, to human-level
semantics}.

\paper{lemaignan2010oro}{2010-oro.jpg}{
    \href{https://doi.org/10.1109/IROS.2010.5649547}{\textbf{ORO, a Knowledge Management Module for Cognitive Architectures in
    Robotics}}
    \newline
    \ul{Lemaignan, S.}, Ros, R., Mösenlechner, L., Alami, R., Beetz, M.
    \newline \textit{IEEE IROS} 2010
}{
    One of the very first knowledge base designed and
    integrated in service robots. Pioneering work which played a key role in
    understanding how intelligent robot can represent their
    knowledge to facilitate communication with humans.
}{
    principal investigator; 177 citations to date
}


One of the most significant application of this work has been on \textbf{natural
language understanding}: by relying on ontologies with human-level semantics to
annotate information flows in the robot's decisional
layers~\autocite{lemaignan2013explicit}, \textbf{I have been able to greatly
simplify the creation of a semantic common-ground between the robot and the
human user}, as I show in~\autocite[presented below]{lemaignan2011grounding}
 and in eg~\autocite{ros2010robot}.


\paper{lemaignan2011grounding}{2012-grounding.jpg}{
    \href{https://doi.org/10.1007/s12369-011-0123-x}{\textbf{Grounding
    the Interaction: Anchoring Situated Discourse in Everyday Human-Robot
    Interaction}} 
    \newline
    \ul{Lemaignan, S.}, Ros, R., Sisbot, E. A., Alami, R., Beetz M.
    \newline \textit{International Journal of Social Robotics} 2012
}{
    In this paper, I develop a natural language parser and I show how symbolic knowledge representation can be
    used by robot to ground natural language interactions, also taking into
    account the unique perspective of the human interactor. One of the
    first example of establishing a semantic common-ground between humans and
    service robots.
}{
    principal investigator; 110 citations to date
}


I also integrated this work on knowledge representation into a
much larger \textbf{semantic-aware architecture}. I led the design of this
architecture, into which were integrated a number of additional cognitive and manipulation
capabilities designed by colleagues. To date, the resulting
system~\autocite[presented below]{lemaignan2017artificial} is still \textbf{one of
the very few complete robotic architecture enabling high-level human-robot
interaction}, and the paper has been the most or second-most cited paper of the
\emph{Artificial Intelligence} journal for the past 4 years.

\paper{lemaignan2017artificial}{2017-ai-cover.jpg}{
    \href{https://doi.org/10.1016/j.artint.2016.07.002}{\textbf{Artificial
    Cognition for Social Human-Robot Interaction: An Implementation}}
    \newline
    \ul{Lemaignan, S.}, Warnier, M., Sisbot, E.A., Clodic, A., Alami, R.
    \newline \textit{Artificial Intelligence} 2017
}{
    Landmark article: one of the first complete, semantic-aware, robotic architecture for
    human-robot interaction, including symbolic knowledge representation,
    situation assessment, natural language grounding, task planning, human-aware
    motion planning and execution.
}{
    principal investigator \& coordinator; 210 citations to date
}


\subsection{Social robotics and child-robot interaction}

Building on my research on semantic-aware human-robot
interaction, I shifted my scientific focus to the \textbf{social aspects of the
interaction} from 2012 onwards. I did put a particular emphasis on the \textbf{psycho-social
underpinnings of social interaction with robots}: 
investigation of perspective taking, and what is called 'mentalizing' in
psycho-linguistics (the cognitive ability to model what others know about the
word, a key pre-requisite to interaction). This led to several
publications~\autocite{ros2010which, warnier2012when, lemaignan2015mutual,
dillenbourg2016symmetry} and I was awarded in 2015 a EU H2020 Marie-Sklodovska
Curie fellowship to specifically investigate this question.

I also investigated a range of other psycho-social determinants, with
significant work on anthropomorphism while supervising J. Fink~\autocite{lemaignan2014cognitive,fink2014which,
fink2014dynamics,lemaignan2015youre}, trust and
engagement~\autocite{flook2019impact,lemaignan2015youre,fink2014which,wijnen2020performing},
or social influence~\autocite{irfan2018social, winkle2019effective} (K.
Winkle's PhD).


In parallel to this basic work, I have developed an expertise in real-world field
deployments of social robots, in particular in educational settings: in addition
to numerous lab-based experiments, \textbf{I have led about 15 field studies
over the last 8 years}, in schools~\autocite{hood2015when, jacq2016building,
baxter2015wider, kennedy2016cautious, chandra2015children, senft2018robots, wallbridge2018using},
medical surgeries~\autocite{lemaignan2016learning}, people's
homes~\autocite{mondada2015ranger},
sport facilities~\autocite{winkle2020couch,winkle2020insitu} and
entrainment~\autocite{lemaignan2012roboscopie} venues.

This breadth of experience gives me a \textbf{unique understanding of the
scientific value, as well as the practical and technical challenges, associated
with real-world deployments of interactive robots}, a key aspect of the research
programme.


My work in child-robot interaction is particularly well recognised, with some
highly-cited publications~\autocite[presented below]{hood2015when, lemaignan2016learning},~\autocite{karim2015review,
kennedy2017child, jacq2016building}. My main contribution in this field is
\textbf{a better understanding of the role and importance of
\emph{socio-cognitive} engagement} between the child and the robot. Using psychological mechanisms
like meta-cognition and the \emph{Protégé effect}, I have been able to
demonstrate long-term engagement in a difficult learning task for children with
learning impairments~\autocite{lemaignan2016learning}.

\paper{lemaignan2016learning}{2016-cowriter.jpg}{
    \href{https://doi.org/10.1109/MRA.2016.2546700}{\textbf{Learning by
    Teaching a Robot: The Case of Handwriting}}
    \newline
    \small
    \ul{Lemaignan, S.}, Jacq, A., Hood, D., Garcia, F., Paiva, A., Dillenbourg, P.
    \normalsize
    \newline \textit{Robotics and Automation Magazine} 2016
}{
    Long-term studies with children and therapists, where we \emph{reverse} the
    social role of the robot (it becomes the learner, and the child, the
    teacher) to significantly improve the children' engagement and
    self-confidence.  A highly-cited, landmark contribution to social robotics
    for education.
}{
    principal investigator; 179 citations (incl. associated conf. article)
}



%
%
%\resizebox{\linewidth}{!}{
%\hspace*{-0.5cm}\begin{tabular}{p{1.7cm}p{7cm}p{8cm}}
%
%
%    \vspace{-.20cm}\includegraphics[height=2.2cm]{thumbs/2019-frontiers-chris.jpg} &
%
%    Wallbridge, C., \ul{Lemaignan, S.}, Senft, E., Belpaeme, T.  
%    \newline\href{https://doi.org/10.3389/frobt.2019.00067}{\textbf{Generating
%    Spatial Referring Expressions in a Social Robot: Dynamic vs Non-Ambiguous}}
%    \newline \textit{Frontiers in AI and Robotics} 2019
%    & \small Challenges the common understanding that robots should be
%    unambiguous: we show that ambiguity is often desirable for fluid and natural
%    human-robot interactions.\textbf{\newline[main study supervisor]}  \\
%
%
%    \vspace{-.20cm}\includegraphics[height=2.2cm]{thumbs/2018-underworlds.jpg} &
%
%    \ul{Lemaignan, S.}, Sallami, Y., Wallbridge, C., Clodic, A., Alami,
%    R. 
%   \newline\href{https://doi.org/10.1109/IROS.2018.8594094}{\textbf{\sc
%    underworlds: Cascading Situation Assessment for Robots}}
%    \newline\textit{IEEE IROS} 2018
%
%    & \small A novel representation technique to efficiently
%    represent multiple parallel states of the world, including imaginary ones.
%    This ability is critical to represent spatio-temporal predictions, and to
%    create models of other agents' representations.
%    \textbf{[principal investigator]}\\
%
%
%
%\end{tabular}
%}


\subsection{Data-driven Human-Robot Interaction}

More recently, my research has expanded towards the emerging field of
\emph{data-driven human-robot interaction}. \textbf{I am one of the pioneer of this
field, and I have led several recent efforts, both 'bottom-up' (eg creating
datasets of social interactions) and 'top-down' (eg developing algorithms to
enable data-driven social behaviours)}.

Applying data-driven approaches (including eg deep learning) to social
interactions is notoriously difficult: what a `social interaction' mean is
difficult to properly frame; it mixes low-level social signals with high-level
semantics; it is dynamic in nature; it features a large number of non-observable
or partially-observable parameters (like the cultural background or the
emotional state of the participants); etc.
It is also very hard to study in practice, as there is no such thing as
`reference social interactions' that we could record and from which we could
train generic machine learning algorithms.

My on-going contributions to this field include \textbf{the definition and
design of social tasks practically suitable for scientific
investigation}~\autocite{lemaignan2016towards, lemaignan2018pinsoro}, and the
subsequent \textbf{acquisition of a large open dataset} of playful social interactions between
children and robots~\autocite[presented below]{lemaignan2018pinsoro}, and made
publicly available to the community~\autocite{pinsoro2018}.

 \paper{lemaignan2018pinsoro}{2018-plosone.jpg}{
    \href{https://doi.org/10.1371/journal.pone.0205999}{\textbf{The
    PInSoRo dataset: Supporting the data-driven study of child-robot social
    dynamics}}
    \newline
    \ul{Lemaignan, S.}, Edmunds E. R., C., Senft, E., Belpaeme, T.
    \newline \textit{PLOS ONE} 2018
}{
    A first-in-kind, large scale dataset of 45h+ of child-child and child-robot social
    interactions. Designed with machine learning in mind, this dataset effectively opens up the field
    of data-driven social psychology, with direct applications in AI and social
    robotics.
}{
    principal investigator
}

This dataset made it possible for my student M. Bartlett and myself to discover
that (1) \textbf{the psycho-social \emph{internal state} of a person can be
largely decoded from her/his body language}, and (2)
\textbf{this internal state can be estimated from only three externally
observable characteristics of the interaction: interaction valence, interaction
balance, and user engagement}~\autocite[presented below]{bartlett2019what}.

\paper{bartlett2019what}{2019-frontiers-maddy.jpg}{
    \href{https://doi.org/10.3389/frobt.2019.00049}{\textbf{What Can You See? Identifying Cues on Internal States from the
    Kinematics of Natural Social Interactions}} 
    \newline
    \small Bartlett, M., Edmunds, C. E. R., Belpaeme, T., Thill, S., \ul{Lemaignan, S.} 
    \normalsize
    \newline \textit{Frontiers in AI and Robotics} 2019
}{  
    Investigates how partially hidden `internal states' (like emotions,
    cooperativeness, etc) can be decoded from simple visible cues, like
    skeletons. Also demonstrates that social situations can be described along 3
    simple dimensions: interaction valence, interaction balance, and user engagement.
}{
    main study supervisor
}


Since 2017, I have also worked with my students E. Senft and K. Winkle on
developing new machine learning techniques to teach autonomous action and social
policies to robots. \textbf{Our key scientific discovery is that, with
the appropriate algorithm, the end-users
themselves can quickly teach complex action and social policies to social
robots, leading to full autonomy on selected tasks, while eliciting a high level
of trust and acceptability} (since the end-users taught the robot
themselves). This breakthrough has been published in Science
Robotics~\autocite[presented below]{senft2019teaching} and other major
international journals and conferences (eg~\autocite{senft2017supervised,
winkle2020insitu}), with several more publications under preparation.


\paper{senft2019teaching}{2019-science.png}{
    \href{https://doi.org/10.1126/scirobotics.aat1186}{\textbf{Teaching robots
    social autonomy from in situ human guidance}}
    \newline
    Senft, E., \ul{Lemaignan, S.}, Baxter, P., Bartlett, M., Belpaeme, T.
    \newline \textit{Science Robotics} 2019
}{
    A novel human-in-the-loop machine learning approach
    to implement social autonomy in a robot, with several deployments in UK
    public schools. This is a first-in-kind demonstration of learning autonomous
    action policy in a high dimensional, socially complex,
    environment.
}{
    main study supervisor
}

%\paper{2017-sparc.jpg}{
%    Senft, E., Baxter, P., Kennedy, J., \ul{Lemaignan, S.}, Belpaeme, T.
%    \newline\href{https://doi.org/10.1016/j.patrec.2017.03.015}{\textbf{Supervised
%    Autonomy for Online Learning in Human-Robot Interaction}}
%    \newline \textit{Pattern Recognition Letters} 2017
%}{
%    The mathematical and technical bases of the SPARC
%    paradigm for human-in-the-loop machine learning, showing that
%    high-dimensional problems can be learnt effectively and rapidely thanks to
%    an innovative input feature selection mechanism.
%}{
%    student supervisor; 22 citations
%}


\subsection{Contributions to methodology}

My research also has a significant impact on research \emph{methodology}.
Inspired by the high scientific standards found in eg psychology literature, I
have been a strong and vocal advocate of open-science, experimental
replicability and statistical robustness.

Indeed, and grounded in my extensive fieldwork experience, I have co-authored
several publications on 'meta-science' in HRI:

\begin{itemize}
    \item I evidenced the current methodological weaknesses in HRI, along with
        recommendations to address them~\autocite{baxter2016characterising};

    \item I showed that HRI researchers sometimes overly rely on, and blindly
        trust, questionable (and typically old) results from
        psychology~\autocite{irfan2018social};

    \item I also made the case for a more balanced view of how robots are
        perceived, in particular in educational
        settings~\autocite{kennedy2016cautious}.

\end{itemize}

To support these efforts, I created and shared with the community tools and dataset to develop our
methodological toolkit and ultimately support better science. For instance, I
presented a novel methodology to assess user engagement in real-time, based on
gaze patterns~\autocite{lemaignan2016realtime}. This work received the
Best Methodology Paper award at the IEEE/ACM HRI conference in 2016; we also
created in 2017 a dataset and a set of recommendations to improve speech
recognition for child-robot interaction~\autocite{kennedy2017child}.

Many of my other technical contributions (presented hereafter) have had a
methodological impact for the broader community (eg I created \texttt{morse} in
2012, the first robot simulator for human-robot interaction; I played an
important role in developing and disseminating the Robot Operating System ROS;
etc.)

\subsection{Contributions to techniques and tools}

I have indeed a number of significant technical contribution to the
field. I have always adopted a open-science approach, releasing all of my
software and hardware contributions to the wider community under open-source
licenses.  I list hereafter the most significant of these technical
contributions.

\begin{itemize}
    \item the \texttt{oro} knowledge base~\autocite{lemaignan2010oro} --
        this highly cited work introduces the usage of ontologies (and
        associated techniques like semantic reasoning) in robotics;

    \item the natural language processing with semantic grounding tool
        \texttt{dialogs}~\autocite{lemaignan2011grounding} -- this other
        highly-cited tool demonstrates how natural language and interactive
        semantic learning can be realised by combining semantic reasoning with
        advanced human perception;

    \item the \texttt{MORSE} simulator~\autocite{echeverria2011morse,
        lemaignan2012morse}, one of the very first simulator enabling
        human-robot interaction simulation, and used by tenths of universities
        worldwide since its inception;

    \item the GenoM verifiable software module
        generator~\autocite{mallet2010genom3} -- this tool makes it possible to
        abstractly specify a robotic module, and automatically generate a code
        skeleton whose behaviour can be proven correct;

    \item the Python-based \texttt{pyRobots} asynchronous supervision
        framework~\autocite{lemaignan2015pyrobots} -- adapts in Python some of the concepts
        originally introduced in the URBI language, making it possible to
        easily write asynchronous supervisors for robots using eg ROS;

    \item a high-accuracy 2D localisation method based on structured
        patterns~\autocite{hostettler2016realtime} -- I supervised this work
        in which we attempt to address the difficult issue of high-accuracy
        indoor localisation in complex, highly-occluded environment. Our method,
        which relies on decoding structured patterns placed in the environment,
        allows for sub-mm localisation with very low computational cost (can
        fully run on a microcontroller);

    \item the design and implementation of Cellulo, a novel holonomic and
        back-drivable mobile robot with haptic feedback, designed from the
        ground-up for child-robot
        interaction~\cite{ozgur2016permanent,ozgur2017cellulo};

    \item the 3D situation assessment platform
        \texttt{underworlds}~\autocite{lemaignan2018underworlds} -- this tool
        is a distributed scene-graph, making it possible to maintain a joint
        dynamic 3D model of environment across software modules. It features
        sensor fusion and spatial reasoning capabilities like perspective
        taking;

    \item a new algorithm for interactive reinforcement
        learning~\autocite{senft2017supervised} -- the algorithm, developed by
        my student E. Senft, has enabled for the first to teach a robot both a
        task and a social action policy \emph{while being in use in the field}.
        We were able to show that after a short training phase, the robot was
        able to reach fully autonomy on a complex educative task.
\end{itemize}


\vspace{2em}

In addition to these academic outputs, other significant technical contributions
include:

\begin{itemize}
    \item \textbf{The port to Python3 of the Robot Operating System (ROS)}, the large
        software framework used by the vast majority of the robotics community
        worldwide;
    \item The \href{https://github.com/ros4hri/ros4hri}{\textbf{ROS4HRI}}
        suite of software module to streamline complex human-robot perception
        pipelines (pre-print:~\cite{mohamed2020ros});
    \item a multi-player online game to simulate human-robot interactions, used
        for teaching and research (eg online studies);
    \item The initial support of the widely used Softbank Nao robot to ROS (this
        work was later officially endorsed by Softbank, ex. Aldebaran Robotics),
        as well as the HOAP-3 humanoid robot;
    \item a review of object recognition
        techniques~\autocite{wallbridge2017qualitative};
    \item a         \href{https://github.com/severin-lemaignan/lecture-software-engineering}{number} of
        \href{https://github.com/severin-lemaignan/lecture-intro-programming-for-robotics}{tutorials}
        and \href{https://github.com/severin-lemaignan/git-presentation}{lectures} on \href{https://github.com/severin-lemaignan/ros-presentation}{software engineering for
        robotics}.
\end{itemize}



\vspace{2em}
\section{Contributions to the development of individuals}

\subsection{Research management}

In addition to my scientific contributions, I play an increasingly
important role in managing research.

While at the `AI for Learning' CHILI Lab at EPFL, I created and successfully
led for 2 years the HRI research group, supervising in total
10 students (including 4 PhD students with whom I co-authored a total of 18
papers). Within that short time frame, I established CHILI as an internationally
recognised research lab in robotics for education. 

Next, during my EU Marie Skłodowska-Curie post-doc at Plymouth University, I
further co-supervised 3 PhD students (co-authoring 17 publications with them).

My current role as a permanent Associate Professor in Social Robotics and AI at
the Bristol Robotics Laboratory further acknowledges my leadership, as I am
\textbf{in charge of defining and implementing the lab's research strategy in
human-robot interactions}. I created the Embedded Cognition for Human-Robot
Interactions (ECHOS) research group, that I now co-lead with Pr. Giuliani,
jointly supervising 15+ PhDs and post-docs. Specifically, the ECHOS group covers most
aspects of situated AI for human-robot interaction, and \textbf{my role includes
strategic planning of the group activities, scientific guidance, recruitment of
staff and prospective students, and grant applications}. I also co-supervise
the BRL's Connected Autonomous Vehicles research group (5 students and
post-docs), with the same management role.

\subsection{Supervision of graduate students and postdoctoral fellows}

\begin{tabular}{p{0.17\linewidth}p{0.8\linewidth}}
    \bf 2018 -- 2019 & \textbf{5 post-docs}, \textbf{5 PhDs}, \textbf{8 MSc students}, Bristol Robotics Lab, UWE, UK \\
    \bf 2015 -- 2018 & \textbf{3 PhDs}, Plymouth University, UK \\
    \bf 2013 -- 2015 & \textbf{5 PhDs}, \textbf{5 MSc students}, EPFL, Switzerland \\
    \bf 2012 -- 2013 & \textbf{2 MSc students}, LAAS-CNRS, France \\
\end{tabular}


\subsection{Teaching activities}

\begin{tabular}{p{0.17\linewidth}p{0.8\linewidth}}
    \bf 2019 --  & \textbf{Associate Professor} teaching mainly HRI at postgraduate level, UWE, UK \\
    \bf 2018 -- 2019 & \textbf{Senior Lecturer} teaching mainly HRI at postgraduate level, UWE, UK \\
    \bf 2015 -- 2018 & \textbf{Lecturer} teaching at undergraduate \&
    postgraduate levels (robotics fundamentals, software engineering, human-robot interaction), Plymouth University, UK \\
    \bf 2013 -- 2015 & \textbf{Teaching Assistant} teaching at undergraduate level (Visual Computing), EPFL, Switzerland \\
    \bf 2008 -- 2012 & \textbf{Teaching Assistant} teaching at undergraduate level (programming, databases, ontologies), INSA Toulouse, France \\
\end{tabular}

\vspace{2em}
\section{Contributions to the wider research community}

Since my PhD, I have established strong peer recognition in the field of human-robot interaction
and cognitive robotics. This includes:

\begin{itemize}[noitemsep,topsep=0pt,parsep=0pt,partopsep=0pt]
    \item numerous \textbf{invited talks} at national and international symposiums and
        events (9 invited talks since Jan. 2018, including \textbf{keynotes} at the UK Robotics
and Autonomous Systems 2019 conference, and at the 2018 AAAI Fall Symposium);
    \item invited to \textbf{high-profile editorial roles}: Programme Committee member of the HRI
conference since 2015; editor of Frontiers In Robotics and AI journal; editor or
Programme Committee member of several leading conferences in AI and Robotics
        (RSS, IROS, IJCAI, HAI, AAMAS);
    \item invited member of the UK EPSRC Peer Review College; member of the EU
        H2020 peer review college; invited reviewer for the French, Dutch, Israeli research agencies;
    \item six invitations as external panel member for PhD defenses over the last two years.
\end{itemize}


\subsection{Organisation of scientific meetings}

\begin{tabular}{p{0.1\linewidth}p{0.85\linewidth}}
    \bf 2021 & \textbf{ACM/IEEE Human-Robot Interaction conference}, Student
    Design Competition chair, virtually held \\
    \bf 2020 & \textbf{ACM/IEEE Human-Robot Interaction conference}, 700+ participants, local chair, Cambridge, UK \\
    \bf 2017 & \textbf{ACM/IEEE Human-Robot Interaction conference}, 400+
    participants, alt.HRI chair, Vienna, AT \\
    \bf 2016 & \textbf{2nd Intl. workshop on Cognitive Architecture for Social HRI}, 45 participants, programme chair, Christchurch, NZ \\
    \bf 2014 & \textbf{Intl. workshop on Simulation for HRI}, 35 participants, programme chair, Bielefeld, DE \\
    \bf 2012 & \textbf{Intl. workshop on MORSE and its applications}, 30 participants, programme chair, Toulouse, FR \\
    \bf 2009 & \textbf{Cognitive Sciences’ Young Researchers Conference}, 150 participants, steering committee, Toulouse, FR \\
\end{tabular}

\subsection{Institutional responsibilities}

\begin{tabular}{p{0.17\linewidth}p{0.8\linewidth}}
    \bf 2019 -- & Full member of the EPSRC Peer Review college \\
    \bf 2017-- & EU H2020 member on the Peer Review College \\
    \bf 2019 -- & Head of the Outreach cluster, Faculty of Technology and Environment, UWE, UK \\
    \bf 2019-- & Invited PhD committee examiner (Örebro U., Uppsala U., KTH,
    Bielefeld U., LAAS-CNRS, BRL)\\
    \bf 2018 -- & HRI module co-lead, MSc level, University of the West of England, UK  \\
    \bf 2017 -- 2018 & Module leader, Robotics fundamentals (undergraduate level), University of Plymouth, UK \\
\end{tabular}

\subsection{Editorial activities}

\begin{tabular}{p{0.17\linewidth}p{0.8\linewidth}}
    \bf 2019 --  & Member of the Robotics, Science and System (RSS) Programme Committee  \\
    \bf 2018 --  & Editorial board of \emph{Frontiers in AI and Robotics} \\
    \bf 2017 --  & Member of the IJCAI Programme Committee  \\
    \bf 2015 -- 2020 & Member of the IEEE/ACM HRI Programme Committee \\
    \bf 2017 -- 2019 & Member of the IEEE IROS Programme Committee  \\
    \bf 2017 -- 2018 & Member of the HAI Programme Committee  \\
\end{tabular}

\vspace{2em}
\section{Contributions to the broader society}

I \textbf{actively engage with policy makers, at national and European
level}: for instance, over the past 2 years, I have been directly interacting
(through participating to panels, visits and one-to-one discussions) with the EU
Research Executive Agency (MSCA AI Cluster 2019); the UK minister for Business,
Energy and Industrial Strategy Greg Clark; the UK minister for Universities,
Science, Research and Innovation Chris Skidmore; the chair of the West of
England authority Tim Bowles; the UK Research \& Innovation Portfolio
manager for Robotics Clara Morri.

I have a \textbf{strong track record of tech transfer}, through patenting (US patent
US20190016213A1) and involvement in national (UK) and EU-level projects focused on
tech-transfer (InnovateUK ROBOPILOT, CAPRI, CAVForth; EU Terrinet, SABRE).

Finally, I actively engage in \textbf{research communication}: my past research has been
covered several times by mainstream international media, including press
releases by Reuters, Press Association; TV coverage by the BBC, Sky News; radio
interviews and broadcast. My academic website (\url{academia.skadge.org})
showcases this media coverage. I also maintain an active, science-focused,
presence on the social media (Twitter handle: @skadge).

\subsection{Policy making}

\begin{tabular}{p{0.17\linewidth}p{0.8\linewidth}}
    \bf 2020 -- & {\bf Expert Collaborator for the European Joint Research Centre} contributing to the UNICEF Guidelines for Responsible Child-Robots Interactions \\
    \bf 2019  & {\bf Invited panel by the EU Research Executive Agency} at the 2019 MSCA AI Cluster, sharing expertise in Human-Robot Interaction \\
\end{tabular}

\subsection{Technology transfer}
\begin{tabular}{p{0.17\linewidth}p{0.8\linewidth}}
    \bf 2018 -- & Co-I on UKRI InnovateUK projects ROBOPILOT, CAPRI, CAVForth, involving direct transfer of technology for automated verification of autonomous vehicles \\
    \bf 2018 -- & Scientific advisor for KickSum Ltd., in the frame of the EU-funded SABRE project \\
    \bf 2018  & Co-inventor on US patent US20190016213A1 on back-driveable, haptic locomotion for small robots \\
\end{tabular}

\subsection{Selected outreach and public dissemination}

\begin{tabular}{p{0.17\linewidth}p{0.8\linewidth}}

    \bf 2019--& Cluster Lead for STEM outreach, University of the West of England \\
    \bf 2019--& Scientific advisor for the Bristol's Science Centre \\
    \bf 2019 & Hosted \href{https://share.coveragebook.com/b/6d7defc7c4a49e93}{large media event} for the Couch25K study~\cite{winkle2020insitu} \\
    \bf 2016--& UK \& EU Robotics Weeks coordinator, University of Plymouth, University of the West of England \\
    \bf 2015 & Hosted large media event for the CoWriter study~\cite{lemaignan2016learning} (coverage by Reuters, BBC Arabic, FastCompany) \\
    \bf 2011& 'Roboscopie' Human-Robot public theater performance, Science Day'11 \url{http://bit.ly/1LQpNWA} \\
    \bf 2008--2011& Toulouse's Cognitive Sciences Students Association, Co-chair \\
    \bf 1997--2012& Executive Committee \& Head of Educational Robotics, Planète Sciences (including coordination of the \textit{EUROBOT} Robotic Competition) \\

\end{tabular}
\vspace{2em}


%\section{MAJOR COLLABORATIONS}

%Name of collaborators, Topic, Name of Faculty/ Department/Centre, Name of University/ Institution/ Country

%%%%%%%%%%%%%%%%%%%%%%%%%%%%%%%%%%%%%%%%%%%%%%%%%%%%%%%%%%%%%%%%%%%%%%%%%%%%%%%%%%%%%%%%%%%%%%%%%%%%%%%%%%%%
%\section{}

%%%%%%%%%%%%%%%%%%%%%%%%%%%%%%%%%%%%%%%%%%%%%%%%%%%%%%%%%%%%%%%%%%%%%%%%%%%%%%%%%%%%%%%%%%%%%%%%%%%%%%%%%%%%
%\section{}

%%%%%%%%%%%%%%%%%%%%%%%%%%%%%%%%%%%%%%%%%%%%%%%%%%%%%%%%%%%%%%%%%%%%%%%%%%%%%%%%%%%%%%%%%%%%%%%%%%%%%%%%%%%%
%\section{}

