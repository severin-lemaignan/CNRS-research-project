
\vspace{3em}
\section{Importance, interdisciplinarity and conclusion}



\subsection{National and International Importance}

This research project addresses the question of how to design socially
assistive robots that are both effective autonomous social agent, and useful,
acceptable and responsible vis-à-vis their end-users.

From an academic perspective, France and the European Union currently enjoy a
2-3 years leadership on research and deployment of socially interactive robots,
mainly built through the several large-scale European projects on that topic,
which took place over the last decade. The CNRS did
play a key role in several of these projects, eg FP6-Cogniron, FP7-CHRIS,
H2020-Spencer, H2020-MuMMer, and has built a solid reputation. It is now
critical that this expertise is maintained and further developed, as to ensure
continued future academic leadership of the CNRS, France and the European Union
in this fast developing, socially critical, domain.

Indeed, surprisingly few groups worldwide have achieved full autonomy for a complex
social robot, the LAAS-CNRS being such a rare examples. \textbf{By joining the CNRS, I
will create the conditions to 'future-proof' this scientific know-how, both
consolidating our expertise, and leading a wide-ranging set of new research
directions on social robotics}. I will as such contribute to further
\textbf{assert scientific leadership on these socially-transformative technologies}.

In addition, my project would create the opportunity for France and Europe to
establish themselves at the forefront of the emerging research on the complex
ethical questions arising from the development of social robots. Indeed, my
research will significantly contribute to the pressing issues around Responsible
AI applied to robotics: the creation of the High-level Expert Group on
Artificial Intelligence by the European Union, and the subsequent release in
2019 of their \emph{Ethics guidelines for trustworthy AI}, evidences the
importance of framing and defining the adequate policies to enable and support
the future development of a safe and trustworthy AI. It however does not address
all of the emerging challenges specifically raised by social robots.

My work will in effect pave the way for similar guidelines to be extended to
social robotics, eg, \emph{embodied, physical} AI. In line with the Europe
Union's strong societal values, the project specifically addresses and frames
the ethical underpinnings of social robots and deliver the guidelines that we
need to inform our future policies on social robotics. Combined with
beyond-state-of-the-art technological developments, \textbf{this research
programme will make a major contribution in securing a safe and responsible
digital future in France, the European Union, and beyond}. 


\subsection{Interdisciplinary nature of the research programme}

Grounded in both the psycho-social literature of human cognition, and the latest
technological advances in artificial cognition and human-robot interaction, the
project delivers major conceptual, technical and experimental contributions
across several fields: AI, ethics, sociology of technology, socio-cognitive
psychology, intelligent robotics, learning technology. As such, \textbf{my
research project builds bridges across multiple disciplinary boundaries}.

I intend to deliver this programme by building on a range of multidisciplinary
methods, including user-centered design; ethnographic and sociological
investigation; expressive non-verbal communication taking inspiration from the
creative arts; embodied cognition; symbolic AI; neural nets and
sub-symbolic AI; interactive machine learning.

Accordingly, I intend to significantly extend the \textbf{strong
interdisciplinary links} that have already been established in the Human-Robot
Interaction groups at both LAAS-CNRS and ISIR. I will seek funding to recruit
PhDs and post-docs with diverse backgrounds in sociology of technology,
cognitive modeling, machine learning, cognitive robotics. Additional expertise
will be sought through academic collaborations, that would include the creation
of an Ethics of HRI working group, specific collaboration with researchers and
practitioners in education and psychology to provide expertise in learning
technologies and cognitive impairment; collaborations with artists (dancers,
sound artists) to provide additional expertise and insights on expressive
communication. Collaborations with local institutions (eg science charities,
schools, hospitals) will complete the \textbf{open and interdisciplinary culture
that I aim to foster} within the host laboratory.


\subsection{Conclusion: my scientific vision}

My research programme is ambitious, both in the short term, and in the longer
run: \textbf{I will lead the design, implementation and real-world demonstration
of socially-intelligent robots. My aim is to create, sustain and better
understand the dynamics of responsible long-term social human-robot
interactions, in order to build robots that have an effective, demonstrable
social utility, while enjoying long-term acceptance by their end-users}.


\textbf{In the short term} (next 5 to 7 years), I will bring together two emerging AI
paradigms (teleological architectures and human-in-the-loop machine learning); I
will integrate them into a state-of-the-art cognitive architecture for
autonomous social robots, relying on multidisciplinary approaches where
relevant; I will create the conditions for a unique, large-scale,
`public-in-the-loop' participatory design approach that will transform how we
think about public engagement with robotic design; finally, I will co-design
and deploy an autonomous robot in real-world, highly social settings,
demonstrating social usefulness and acceptance over a significant period of
time.

\textbf{In the longer run}, I will drive, locally, nationally and internationally, the
future of intelligent social robots. Both from a societal point of view (what
role for robots in our society? how to involve the general public into the
design and implementation of these robots?  how to ensure the technology is
inclusive? what ethical framework?) and from a technological point of view (what
models of the humans and their environment do we need to build? what algorithms
for robots to `learn by doing' and become `good citizens' in our digital
society?)

These two facets (societal progress and technical progress) should always go
hand-in-hand: societal research needs to fully understand the limits and
opportunity of the underlying technology; technology must be framed
by societal needs and ethical considerations. I believe my cross-disciplinary
academic profile and extensive experimental and technological experience provide me
with the skills and understanding required to successfully progress them both.

\begin{framed}

\bf

At its core, my research programme is about co-defining, co-designing and
building the social robots of tomorrow: it offers a vision of AI and social
robotics that places the human at the centre of these emerging technologies,
to foster novel social dynamics that are acceptable and beneficial to
society.

I propose an ambitious yet realistic scientific and technical pathway to
progress toward this major scientific endeavour. By joining the CNRS, I will
bring in a unique expertise and I will lead the creation of autonomous
social robots that not only learn social behaviours with and from the public
and end-users, but that are also co-designed from the ground-up to be
acceptable, responsible and useful to the humans they will serve.

%In other words, \project seeks to answer \textbf{why
%would we want to embed social robots in our society?}, and \textbf{how to do
%so?}, from a technology and Responsible AI perspective.

\end{framed}

%\begin{wrapfigure}[9]{r}{0.35\linewidth}
%    \centering

