\chapter{Importance and Integration in the scientific landscape}

\section{National and International Importance}

This research project addresses the questions of how to design socially
assistive robots that are both effective autonomous social agent, and useful,
acceptable and responsible vis-à-vis their end-users.

{\color{gray} \TODO{Update that section for France/CNRS/Europe/ANITI}\\
These questions are of prime societal importance, and this research closely
aligns with the \textbf{EPSRC Delivery Plan \emph{Connected Nation} and
\emph{Healthy Nation} priorities}. Specifically, the project
investigates and will significantly advance the questions of \ul{Trustworthy
autonomous AI}, \ul{Multidisciplinary approaches to technology acceptability}
and \ul{Technology for the public good}\footnote{EPSRC Delivery Plan 2019:
\url{https://epsrc.ukri.org/about/plans/dp2019/}}.

The project is also closely aligned with UKRI Healthcare Technology Grand
Challenge: \emph{Transforming Community Health and
Care}\footnote{https://epsrc.ukri.org/research/ourportfolio/themes/healthcaretechnologies/strategy/grandchallenges/}
by significantly advancing our capabilities in term of socially assistive
robotics.

More broadly, and as a multidisciplinary project, \project relates to several
themes of the EPSRC portfolio. The main ones are: \emph{Human-computer
interaction} and \emph{Social computing/interactions} within the \emph{Digital
Economy} theme, \emph{Assistive technology} within the \emph{Healthcare
technology} theme, and \emph{Artificial Intelligence} and \emph{Robotics} within
the Engineering theme.
}

From an academic perspective, the UK and the European Union currently enjoy a
2-3 years leadership on research and deployment of socially interactive robots
(mainly built through the several large-scale European projects on that topic,
which took place over the last decade). The UK did play a key role in several of
these projects (eg FP6-Cogniron, FP7-CHRIS, FP7-STRANDS, FP7-Poeticon++), and
has built a solid reputation. It is now critical that this expertise is
maintained and further developed, as to ensure the future academic leadership of
the UK.

In addition, my project would create the opportunity for France and Europe to establish
themselves at the forefront of the emerging research on the complex ethical
questions arising from the development of social robots. Indeed, my research
will significantly contribute to the pressing issues around Responsible AI
applied to robotics: the creation of the High-level Expert Group on Artificial
Intelligence by the European Union, and the subsequent release in 2019 of their
\emph{Ethics guidelines for trustworthy AI}, evidences the importance of framing
and defining the adequate policies to enable and support the future development
of a safe and trustworthy AI. It however does not address any of the emerging
challenges raised by social robots.

My work will in effect pave the way for similar guidelines to be extended to
social robotics, eg, \emph{embodied, physical} AI. In line with the UK's strong
societal values, the task T1.1, which continues throughout the project, will
specifically address and frame the ethical underpinnings of social robots
and deliver the guidelines that we need to inform our future policies on social
robotics. Combined with beyond-state-of-the-art technological developments,
\textbf{this research programme will make a major contribution in
securing a safe and responsible digital future in France, the European Union, and beyond}. 


\section{Interdisciplinary nature of the research programme}

\project paves the way for a better understanding of the societal challenges
raised by the rapid development of AI and robotics. Grounded in both the
psycho-social literature of human cognition, and the latest technological
advances in artificial cognition and human-robot interaction, the project
delivers major conceptual, technical and experimental contributions across
several fields: AI, ethics, sociology of technology, intelligent robotics,
learning technology. As such, \textbf{\project builds bridges across
multiple disciplinary boundaries}.

\project delivers this programme by building on a range of multidisciplinary
methods, including user-centered design; ethnographic and sociological
investigation; expressive non-verbal communication, including dance and
puppetering; embodied cognition; symbolic AI; neural
nets and sub-symbolic AI; interactive machine learning.

Accordingly, the project builds on a \textbf{strong interdisciplinary team}: the
post-docs directly recruited on \project will have backgrounds in sociology of
technology (PD1), cognitive modeling (PD2), machine learning (PD3), cognitive
robotics (PD4). Additional expertise will be recruited to provide specific
support: the \project Ethics Advisory Board will contribute expertise to guide
the work on ethics; Dr. Newbutt will provide expertise in learning technologies
and cognitive impairment; Dr. Meckin will provide expertise in sound-based
expressive communication; the WeTheCurious science centre will provide training in
large-scale public engagement; the Bristol Children's hospital will bring the
required expertise in working with young patients; the RustySquid company will provide expertise in
expressive arts and puppetering.

\section{Integration with the local research landscape}

\TODO{TODO, in particular connections to ANITI}

