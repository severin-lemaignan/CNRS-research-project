
\vspace{3em}
\section{Importance, ambition and outlook}


\subsection{National and International Importance}

This research project addresses the questions of how to design socially
assistive robots that are both effective autonomous social agent, and useful,
acceptable and responsible vis-à-vis their end-users.

From an academic perspective, France and the European Union currently enjoy a
2-3 years leadership on research and deployment of socially interactive robots,
mainly built through the several large-scale European projects on that topic,
which took place over the last decade. The CNRS (and the LAAS in particular) did
play a key role in several of these projects, eg FP6-Cogniron, FP7-CHRIS,
H2020-Spencer, H2020-MuMMer, and has built a solid reputation. It is now
critical that this expertise is maintained and further developed, as to ensure
continued future academic leadership of the CNRS, France and the European Union
in this fast developing, socially critical, domain.

In addition, my project would create the opportunity for France and Europe to
establish themselves at the forefront of the emerging research on the complex
ethical questions arising from the development of social robots. Indeed, my
research will significantly contribute to the pressing issues around Responsible
AI applied to robotics: the creation of the High-level Expert Group on
Artificial Intelligence by the European Union, and the subsequent release in
2019 of their \emph{Ethics guidelines for trustworthy AI}, evidences the
importance of framing and defining the adequate policies to enable and support
the future development of a safe and trustworthy AI. It however does not address
any of the emerging challenges raised by social robots.

My work will in effect pave the way for similar guidelines to be extended to
social robotics, eg, \emph{embodied, physical} AI. In line with the Europe
Union's strong societal values, the project specifically addresses and framees
the ethical underpinnings of social robots and deliver the guidelines that we
need to inform our future policies on social robotics. Combined with
beyond-state-of-the-art technological developments, \textbf{this research
programme will make a major contribution in securing a safe and responsible
digital future in France, the European Union, and beyond}. 


\subsection{Interdisciplinary nature of the research programme}

This research programme paves the way for a better understanding of the societal
challenges raised by the rapid development of AI and robotics. Grounded in both
the psycho-social literature of human cognition, and the latest technological
advances in artificial cognition and human-robot interaction, the project
delivers major conceptual, technical and experimental contributions across
several fields: AI, ethics, sociology of technology, intelligent robotics,
learning technology. As such, \textbf{my research project builds bridges across
multiple disciplinary boundaries}.

I deliver this programme by building on a range of multidisciplinary methods,
including user-centered design; ethnographic and sociological investigation;
expressive non-verbal communication, including dance and puppetering; embodied
cognition; symbolic AI; neural nets and sub-symbolic AI; interactive machine
learning.

Accordingly, I intend to significantly extend the \textbf{strong
interdisciplinary links} that have already been established in the Human-Robot
Interaction group of LAAS-CNRS. I will seek funding to recruit PhDs and
post-docs with backgrounds in sociology of technology, cognitive modeling,
machine learning, cognitive robotics. Additional expertise will be sought
through academic collaborations: the creation of an Ethics of HRI working group
will contribute expertise to guide the work on ethics; specific collaboration
with researcher in education and psychology will provide expertise in learning
technologies and cognitive impairment; collaborations with artists (dancers,
sound artists) will provide additional expertise and insights on expressive
communication; and collaboration with local institutions (eg science charities,
schools, hospitals) will complete the \textbf{open and interdisciplinary
culture} that I aim to foster within the laboratory.


\subsection{Concluding remark: the key scientific aims of my research programme}

\TODO{TBD, based on overview at begining of project}

As outlined in this document, my research project is about designing and
delivering a ground-breaking embodied AI for socially intelligent robots, with
long-term social utility and demonstrated acceptance in the real world.

This breakthrough is made possible by a combination of novel methodologies and
the principled integration of complex socio-cognitive capabilities:

\begin{itemize}
        \item crowd-sourced social interaction patterns;
        \item `public-in-the-loop' machine learning;
        \item a novel spatio-temporal and social model of the robot's environment;
        \item novel, non-repetitive, social behaviour production based on
            generative neural networks;
        \item and finally, an integrative cognitive architecture, driven by
            long-term social goals.
\end{itemize}


In addition, I will deliver the conceptual and ethical framework
required to further support the public debate and policy making process
around social robots, and concretely demonstrate lifescale applications of
this technology with ambitious, long-term deployments of autonomous robots
in high impact, social environments.

The Laboratoire d'Analyse et d'Architecture des Systèmes (LAAS), part
of the Artificial and Natural Intelligence Toulouse Institute (ANITI), would
be an ideal host laboratory to successfully conduct this programme: its
strong track-record in autonomous interactive robots, combined with the breadth
of expertise available within the ANITI institute, would prove instrumental in
scaffolding and accelerating several of the key science breakthrough I target
with this project.

Closely aligned with the national and European research priorities,
this research project creates a excellent opportunity to assert the CNRS and
Europe as worldwide leaders in Social and Intelligent Robotics.


This research project is ambitious, and I believe I am in a unique position to
deliver on its work plan. I already have established international recognition in
human-robot interaction and have likewise demonstrated strong leadership by
leading research teams in three different institutions. As presented in my
track-record, breadth of my interdisciplinary research covers the scientific expertise
required by the project, providing me with a unique overall perspective and
understanding of the domain. I am also a technology expert, with major software
and hardware contributions to the robotic community. As such,
I have a good grasp of the technical feasibility of the proposed work.

This research programme is indeed ambitious, with an experimental programme that
goes significantly beyond the state of the art. It will provide a lasting
scientific and technical legacy, and will inject a new momentum into the strong
human-robot expertise of LAAS-CNRS. Finally, this research programme would also
be a powerful enabler: it creates the opportunity to establish myself and the
CNRS as world-leader in the emerging field of socially-driven, responsible
autonomous robots, significantly reinforcing the national and European capacity
in this critical field for our digital future.

