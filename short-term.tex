\section{Research plan for the first five years}
\TODO{Revisit this section -- do we need it? if so, needs rework}

\textbf{T5.1: A robot companion to support physical, mental and social
well-being in SEN schools}

Inspired by a similar large-scale deployment of social robots in Hong-Kong's SEN
schools\cite{robot4sen}, the first study investigate whether a socially
assistive robot can effectively support the development, social
interactions and well-being of children with a long-term mental condition. This
study will take place within the network of Bristol-based SEN schools, with
which I already have an on-going collaboration.  Specifically, the two main
questions we seek to investigate are: What are the social underpinnings of the
successful integration of a social robot in the school ecosystem? Can ambitious
co-design with the end-users (teachers) deliver a `net gain' for the learning,
social interaction and well-being of the students? 

The core of the study consists in deploying the R1 social robot in one of
Bristol-based SEN school (Mendip Primary School, with possible extensions to
other schools), to investigate how the robot can help shaping a social school
ecology that fosters mental well-being, while effectively supporting teachers
and students in their learning. 

The study will adopt a strong participatory design approach, inspired by
Patient and Public Involvement methodologies (PPI\cite{boivin2010patient}),
with 3 one-day focus groups organised with the school teachers; two evening focus group with the
school parents, prior to the study; and several preparatory workshop at the
school premises to involve the students as well.

%During the first
%workshop, the teachers will be introduced to the robot capabilities with
%examples of robot-supported teaching activities, and the robot's visual
%programming interface will be introduced. We will also conduct group discussions
%on how the robot can best be integrated in the daily school routine and
%classroom context. During the second workshop, the teachers will be invited to
%create novel activities, with the support of the research team. An evening focus
%group will be organised as well with the parents, to integrate their
%perspectives in the design of the robotic system.  Will we formally analyse the
%data from this – will it become a research paper? 

%Following the workshops, the teacher-oriented codesign of the robot's activities
%and supervision tools (eg to start/stop/pause/resume activities) will be
%finalised and implemented by the research team.

The school study itself will take place during Y3, with the robot permanently
based at the school. The robot will take part in the
regular teaching and other daily routines of the school, and will directly
interact with the children, learning its action policy (`when to do what') from
initial co-design with the teachers, followed by progressive in-situ teaching (see
T4.2).

During selected `observation days', observations will be conducted by the
research team, and regular semi-structured interviews will be conducted with the
teachers, parents, and where possible, the children themselves (using engagement
metrics like the Inclusion of Other in Self task and Social-Relational
Interviews\cite{westlund2017measuring}), to understand how the robot impacts
the school dynamics  (both positively and potentially negatively).

The task will be jointly supervised with local colleague and expert Dr.Nigel Newbutt,
who has a long track record of working with special needs schools.

\textbf{T5.2 -- A robot companion to support isolated children during their
hospital stay}

The second experiment will take place within the paediatric ward for long-term
conditions at the Bristol Children's Hospital. The ward has 8 beds, with
children staying from a few weeks to several years. Over the course of the
one-year deployment, we expect the robot to interact with about 30 children,
their parents, and the hospital staff (nurses, doctors).

Similar to the first experiment, we will be using a \emph{mutual shaping}
approach\cite{winkle2018social} to design the role of the robot with the
different stakeholders (nurses, doctors, parents, children), in order to
experimentally investigate how a social robot can support hospitalised children
with long-term conditions. The robot's role will revolve around facilitating
social interactions between (possibly socially isolated) children, by fostering
playful interaction within the paediatric ward.

This second experiment complements the first one by evidencing the commonalities
and divergences in terms of social interactions when the robot is moved to a
different environment. While the hospital eco-system is comparatively smaller that the SEN school one,
people `live' at the ward day and night; it becomes \emph{de facto} the second home of the
children, and the children will have more interaction opportunities than at the
SEN school (where the robot is shared amongst a larger group). As a consequence,
we expect to observe different interaction patterns, with potentially deeper
affective engagement between the robot and the other ward's `inhabitants'.
Specific safeguarding measures will be put in place with the hospital team, and
resulting observations will feed into the ethical guidelines of T1.1.

%%%%%%%%%%%%%%%%%%%%%%%%%%%%%%%%%
%% Gantt chart

%\begin{landscape}
%\begin{figure}[!ht]
%\resizebox{\linewidth}{!}{
%    \input{gantt}
%}
%\end{figure}
%\end{landscape}


