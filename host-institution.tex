\section{Proposed host laboratories}

\subsection{Laboratoire d'Analyse et d'Architecture des Systèmes (LAAS) -- UPR 8001}

The Laboratoire d'Analyse et d'Architecture des Systèmes (LAAS), part of the
\emph{Artificial and Natural Intelligence Toulouse Institute} (ANITI), has a
long and strong track-record in autonomous interactive robots. Combined with the
breadth of expertise in AI available within the Toulouse-wide ANITI institute,
it would effectively support and accelerate several of the key science
breakthrough I target with this project.

Indeed, the LAAS is one of the few research group worldwide that has achieved
full autonomy for complex socio-cognitive robots
(eg~\cite{lemaignan2017artificial}).  This has been made possible by the
long-term commitment of the laboratory to cognitive robotics, spearheaded by Pr.
Rachid Alami. Over the last 20 years, Alami and colleagues have indeed built a
large, ambitious body of research, enabling a wide range of cognitive skills on
service robots. To cite a few that would be directly relevant to my research
programme: reasoning languages~\autocite{Ingrand1996} and
supervision~\autocite{Clodic2009}; 3D motion and task
planning~\autocite{Sisbot2008, Mainprice2011}; human-aware symbolic task
planning~\autocite{alami2005task,Alili2008,Lallement2014,milliez2016using};
cognitive architectures~\autocite{devin2016implemented}.

The LAAS also has a long-standing commitment to technical excellence. This has
enabled a large number of technically challenging experimental deployments of
robots, including social and interactive robots (eg~\textcite{alami2005task}).
This expertise would directly benefit my research programme as well.

My research programme links directly to these themes, and focuses additionally
on (1) social AI, (2) data-driven human-robot interaction, and (3) the
real-world societal impact of human-robot interactions. Those emerging research
areas are not yet developed at LAAS; my affectation there would bring in this novel
expertise.

At the local level, my interdisciplinary research programme would fits
especially well in the ANITI agenda (one of the four French Institute for AI;
200+ researchers in the Toulouse region): as put by Nicholas Asher, ANITI's
director, while discussing with him, one of the ANITI's challenges is to
identify powerful transverse applications that would showcase the integration of
the range of AI techniques developed within the institute into a complete
system. The robots I will develop in my programme would be natural candidates
for such integration, and the diverse research conducted within ANITI (eg work on Fair \& Robust Machine
Learning (Jean-Michel Loubès), Social Trust in AI (Céline Castets-Renard), AI
for policy learning (Cesar Hidalgo, Leïla Amgoud), or AI for complex behaviour
generation (Nicolas Mansard)) would undoubtedly accelerate the research and
increase its impact.

%\subsubsection{Responsible AI}
%
%
%\begin{itemize}
%    \item Fair \& Robust Machine Learning
%        \url{https://aniti.univ-toulouse.fr/chaire-jean-michel-loubes/}
%Jean-Michel Loubès
%
%\item Law, Accountability and Social Trust in AI
%    \url{https://aniti.univ-toulouse.fr/chaire-celine-castets-renard/}
%Céline Castets-Renard
%
%\item Moral AI
%    \url{https://aniti.univ-toulouse.fr/chaire-moral-ai/}
%Jean-François Bonnefon
%
%\item New certification approaches of critical AI based systems 
%    \url{https://aniti.univ-toulouse.fr/chaire-claire-pagetti/}
%Claire Pagetti
%
%\end{itemize}
%
%\subsubsection{AI for policy learning}
%
%\begin{itemize}
%    \item Augmented Society
%        \url{https://aniti.univ-toulouse.fr/chaire-cesar-hidalgo/}
%        Cesar Hidalgo
%    \item Empowering Data-driven AI by Argumentation and Persuasion
%        \url{https://aniti.univ-toulouse.fr/chaire-leila-amgoud/}
%Leïla Amgoud
%\end{itemize}
%
%\subsubsection{AI for complex behaviour generation}
%
%\begin{itemize}
%    \item Motion Generation for Complex Robots 
%        \url{https://aniti.univ-toulouse.fr/chaire-nicolas-mansard/}
%Nicolas Mansard
%\end{itemize}

\subsection{Institut des Systèmes Intelligents et de Robotique (ISIR) -- UMR 7222}

\TODO{TODO Section}
