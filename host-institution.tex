\section{Adequation of the host laboratory to the research project}

The \textbf{Laboratoire d'Analyse et d'Architecture des Systèmes} (LAAS), part
of the \textbf{Artificial and Natural Intelligence Toulouse Institute} (ANITI), would
be an ideal host laboratory to successfully conduct my research programme: its
strong track-record in autonomous interactive robots, combined with the breadth
of expertise available within the ANITI institute, would prove instrumental in
scaffolding and accelerating several of the key science breakthrough I target
with this project.

\subsection{LAAS expertise}

The LAAS is one of the few research group worldwide that has achieved full
autonomy for complex socio-cognitive robots (eg~\cite{lemaignan2017artificial}).

This has been made possible by the long-term commitment of the laboratory to
cognitive robotics, spearheaded by Pr. Rachid Alami. Since the mid-90's, Alami
and colleagues have indeed build a large, ambitious body of research, enabling a
wide range of cognitive skills on service robots. To cite a few: reasoning
languages~\cite{Ingrand1996}; symbolic
reasoning~\cite{lemaignan2010oro} and supervision~\cite{Clodic2009}; 3D motion
and task planning~\cite{Sisbot2008, Mainprice2011}; human-aware symbolic task
planning~\cite{Alili2008,Lallement2014,milliez2016using}; cognitive
architectures~\cite{lemaignan2017artificial,devin2016implemented}; language
understanding~\cite{lemaignan2011grounding}.
These achievements are underpinned by excellent science, and also the strong
commitment to technical excellence (eg~\cite{mallet2010genom3}), an often
overlooked yet unique strength of the LAAS.


I took part myself to this scientific journey, from 2008 to 2012, contributing
key cognitive elements (symbolic reasoning, perspective taking, language
understanding), as well as integrating the research into a meaningful, coherent
architecture~\cite{lemaignan2017artificial}.

These four years spent at LAAS,
followed by 8 years in other institutions abroad put me in the unique position
of both understanding in great detail and fully appreciating the approach and
strengths of the LAAS, while also being able to appraise the potential
weaknesses and direction of change.

I would like to outline here some

By joining the laboratory, I would create the conditions to
'future-proof' this scientific know-how, while developing a wide-ranging set of
new research directions that promise to have a transformative impact on our
digital future.

\newpage
\subsection{Integration within the broader local research landscape}\label{collaborations}

\TODO{TODO, in particular connections to ANITI}
\TODO{Here, give a high-level overview of the potential for integration with
existing structures; detailled collaboration opportunities to be specified
earlier in the project.}

- verbal behaviours -> collaboration within ANITI?
- collaboration with Rafaelle

- experimental work: partnerships/institution in Toulouse


\subsubsection{ANITI themes}

ANITI, the \emph{Artificial and Natural Intelligence Toulouse Institute} is one
of the four French Institutes for AI, aiming at international leadership in AI.
It brings together 200+ research scientists around Toulouse, and organises the
research around three main programmes: Acceptability and AI; Certifiable AI;
Collaborative AI.

My research programme aligns particularly well with these themes. I would
directly benefit of excellent collaboration opportunities (see below), and in
return, my programme would create unique transversal opportunities to tie
together key sub-themes.

Specifically, the 

bridge between 

\subsubsection{Responsible AI}


\begin{itemize}
    \item Fair \& Robust Machine Learning
        \url{https://aniti.univ-toulouse.fr/chaire-jean-michel-loubes/}
Jean-Michel Loubès

\item Law, Accountability and Social Trust in AI
    \url{https://aniti.univ-toulouse.fr/chaire-celine-castets-renard/}
Céline Castets-Renard

\item Moral AI
    \url{https://aniti.univ-toulouse.fr/chaire-moral-ai/}
Jean-François Bonnefon

\item New certification approaches of critical AI based systems 
    \url{https://aniti.univ-toulouse.fr/chaire-claire-pagetti/}
Claire Pagetti

\end{itemize}

\subsubsection{AI for policy learning}

\begin{itemize}
    \item Augmented Society
        \url{https://aniti.univ-toulouse.fr/chaire-cesar-hidalgo/}
        Cesar Hidalgo
    \item Empowering Data-driven AI by Argumentation and Persuasion
        \url{https://aniti.univ-toulouse.fr/chaire-leila-amgoud/}
Leïla Amgoud
\end{itemize}

\subsubsection{AI for complex behaviour generation}

\begin{itemize}
    \item Motion Generation for Complex Robots 
        \url{https://aniti.univ-toulouse.fr/chaire-nicolas-mansard/}
Nicolas Mansard
\end{itemize}

